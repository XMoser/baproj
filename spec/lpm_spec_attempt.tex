\documentclass{article}
\usepackage{amsmath}
\usepackage{amssymb}
\DeclareMathOperator{\initialize}{initialize}
\DeclareMathOperator{\update}{update}
\DeclareMathOperator{\delete}{delete}
\DeclareMathOperator{\matchlength}{matchLength}
\DeclareMathOperator{\lookup}{lookup}
\begin{document}
\section{Data structures}
\subsection{Key}
A key is the primary element used to identify entries in the lpm table. A key
is represented as follows:
\[
    Key: K(P, l),
\]
where $l$, the prefix length, is a non-negative integer and $P = p_1p_2...p_l$,
the prefix, is a string of symbols.
\subsection{Table}
A Table stores keys with the values associated to them. It is represented as
follows:
\[
    Table: T(E, m),
\]
where $m$, the maximum number of entries in the table, is a positive integer and
\[
    E = \{(K_1, v_1), (K_2, v_2), ...\}
\]
is the collection of entries, where the $K_i$s are keys and the $v_i$s are the
integer values associated with those keys.
\section{Functions}
\subsection{initialize}
The $\initialize$ function creates a new empty table with a given maximum number
of entries. It is specified as follows:
\[
    \initialize(m) = T(E, m),
\]
where $m \in \mathbb{N} \setminus \{0\}$ and $|E| = 0$.
\subsection{update}
The $\update$ function inserts a new key-value pair in a table or updates the value
associated with a key, if the key given as argument is already part of the
entries in the table. The table given as argument must have space for at least
one more entry. The $\update$ function must not modify table entries that do not
correspond to the key given as argument. It is specified as follows:
\[
    \update(T(E, m), (K, v)) = T(E', m),
\]
where $|E| < m$, $(K, v) \in E'$,
$E' \setminus \{(K, v)\} = E \setminus \{(K, \_)\}$ and
\[
    |E'|=
    \begin{cases}
    |E|+1,   & \text{if } (K, \_) \notin E\\
    |E|,  & \text{otherwise}
    \end{cases}
\]
\subsection{delete}
The $\delete$ function removes the entry corresponding to a specified key from the
table. The key given as argument must belong to the entries of the table given
as argument. The $\delete$ function must not modify entries in the table that do
not correspond to the key given as argument. It is specified as follows:
\[
    \delete(T(E, m), K) = T(E', m),
\]
where $(K, \_) \in E$, $E' = E \setminus \{(K, \_)\}$ and $|E'| = |E|-1.
\subsection{lookup}
lookup is the main function used to perform the actual longest prefix match. It
takes as arguments a non-empty table and a key and returns the value associated
with the table entry that best matches the key given as argument. For the
purpose of specifying the $\lookup$ function, we introduce the $\matchlength$
function, defined as follows:
\subsubsection{matchLength}
The $\matchlength$ function takes two keys as parameters and returns the number of
symbols in the first key matching the prefix in the second key.
\[
    \matchlength(K(P, l), K'(P', l')) = i,
\]
where $i$ is the largest integer such that
\(
p_1p_2...p_i = p'_1p'_2...p'_i,
\)
and $i \leq \min(l, l')$. \newline \newline
With the $\matchlength$ function, we can specify the $\lookup$ function as
follows:
\[
    \lookup(T(E, m), K) = v,
\]
where $|E| > 0$, $(K', v) \in E$, and
\[
    \forall (K'', \_) \in E \text{ such that } K'' \neq K' :
\]
\[
    \matchlength(K, K'') \leq \matchlength(K, K')
\]
\end{document}
