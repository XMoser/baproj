\documentclass{article}
\usepackage{amsmath}
\DeclareMathOperator{\update}{update}
\DeclareMathOperator{\delete}{delete}
\DeclareMathOperator{\matchlength}{matchLength}
\DeclareMathOperator{\min}{min}
\DeclareMathOperator{\lookup}{lookup}
\begin{document}
\section{Data structures}
\subsection{Key}
A key is the primary element used to identify entries in the lpm table. A key
is represented as follows:
\[
    Key: K(D, p),
\]
where $p$, the prefix length, is a non-negative integer and
$D = d_1d_2...d_pd_{p+1}...$, the data,  is a string of symbols.
The sub-string $d_1...d_p$ is the prefix associated with the key $K$. If $p$ is
equal to 0, the prefix is empty.
\subsection{Table}
A Table stores keys with the values associated to them. It is represented as
follows:
\[
    Table: T(E, n),
\]
where $n$, the size of the table, is a non-negative integer.
\[
    E = \{(K_1, v_1), (K_2, v_2), ..., (K_n, v_n)\}
\]
is the collection of entries,
where the $K_i$s are keys and the $v_i$s are the integer values associated with
those keys.
\section{Functions}
\subsection{update}
The $\update$ function inserts a new key-value pair in a table or updates the value
associated with a key, if the key given in argument is already part of the
entries in the table. It is specified as follows:
\[
    \update(T(E, n), (K, v)) = T(E', n'),
\]
where $(K, v) \in E'$ and\newline
\[
    n'=
    \begin{cases}
    n+1,   & \text{if } (K, \_) \notin E\\
    n,  & \text{otherwise}
    \end{cases}
\]
\subsection{delete}
The $\delete$ function removes the entry corresponding to a specified key from the
table. If the specified key is not in the entries of the table, it does nothing.
It is specified as follows:
\[
    \delete(T(E, n), K) = T(E', n'),
\]
where $(K, \_) \notin E'$ and,
\[
    n'=
    \begin{cases}
    n-1,    & \text{if } (K, \_) \in E\\
    n,              & \text{otherwise}
    \end{cases}
\]
\subsection{lookup}
lookup is the main function used to perform the actual longest prefix match. It
takes as arguments a table and a key and returns the value associated with the
table entry that best matches the key given as argument. For the purpose of
specifying the lookup function, we introduce the $\matchlength$ function, defined
as follows:
\subsubsection{matchLength}
The $\matchlength$ function takes two keys as parameters and returns the number of
symbols in the first key matching the prefix in the second key.
\[
    \matchlength(K(D, p), K'(D', p')) = i,
\]
where $i$ is the largest integer such that
\(
d_1d_2...d_i = d'_1d'_2...d'_i,
\)
and $i \leq \min(p, p')$. \newline \newline
With the $\matchlength$ function, we can specify the $\lookup$ function as
follows:
\[
    \lookup(T(E, n), K) = v,
\]
where $(K', v) \in E$, and
\[
    \forall (K'', \_) \in E \text{ such that } K'' \neq K' :
\]
\[
    \matchlength(K, K'') \leq \matchlength(K, K')
\]
\end{document}
