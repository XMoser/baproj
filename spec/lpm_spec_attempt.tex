\documentclass{article}
\usepackage{amsmath}
\usepackage{amssymb}
\DeclareMathOperator{\initialize}{initialize}
\DeclareMathOperator{\update}{update}
\DeclareMathOperator{\delete}{delete}
\DeclareMathOperator{\matchlength}{matchLength}
\DeclareMathOperator{\lookup}{lookup}
\begin{document}
\section{Data structures}
\subsection{Table}
A Table stores prefixes with the values associated to them. It is represented as
follows:
\[
    Table: T(E, m),
\]
where $m$, the maximum number of entries in the table, is a positive integer and
\[
    E = \{(P_1, v_1), (P_2, v_2), ...\}, P_i = p_{i1}p_{i2}p_{i3}...
\]
is the collection of entries, where the $P_i$s, the prefixes are strings of
symbols and the $v_i$s are the integer values associated with those prefixes.
\section{Functions}
\subsection{initialize}
The $\initialize$ function creates a new empty table with a given maximum number
of entries. It is specified as follows:
\[
    \initialize(m) = T(E, m),
\]
where $m \in \mathbb{N} \setminus \{0\}$ and $|E| = 0$.
\subsection{update}
The $\update$ function inserts a new prefix-value pair in a table or updates
the value associated with a prefix, if the prefix given as argument is already
part of the entries in the table. The table given as argument must have space
for at least one more entry. The $\update$ function must not modify table
entries that do not correspond to the prefix given as argument. It is specified
as follows:
\[
    \update(T(E, m), (P, v)) = T(E', m),
\]
where $|E| < m$, $(P, v) \in E'$,
$E' \setminus \{(P, v)\} = E \setminus \{(P, \_)\}$ and
\[
    |E'|=
    \begin{cases}
    |E|+1,   & \text{if } (P, \_) \notin E\\
    |E|,  & \text{otherwise}
    \end{cases}
\]
\subsection{delete}
The $\delete$ function removes the entry corresponding to a specified prefix from the
table. The prefix given as argument must belong to the entries of the table given
as argument. The $\delete$ function must not modify entries in the table that do
not correspond to the prefix given as argument. It is specified as follows:
\[
    \delete(T(E, m), P) = T(E', m),
\]
where $(K, \_) \in E$, $E' = E \setminus \{(P, \_)\}$ and $|E'| = |E|-1.
\subsection{lookup}
lookup is the main function used to perform the actual longest prefix match. It
takes as arguments a non-empty table and a prefix and returns the value
associated with the table entry that best matches the prefix given as argument.
For the purpose of specifying the $\lookup$ function, we introduce the
$\matchlength$ function, defined as follows:
\subsubsection{matchLength}
The $\matchlength$ function takes two prefixes as parameters and returns the
number of symbols in the first prefix matching the symbols in the second prefix.
\[
    \matchlength(P, P') = i,
\]
where $i$ is the largest integer such that
\(
p_1p_2...p_i = p'_1p'_2...p'_i,
\)
and $i \leq \min(|P|, |P'|)$. \newline \newline
With the $\matchlength$ function, we can specify the $\lookup$ function as
follows:
\[
    \lookup(T(E, m), P) = v,
\]
where $|E| > 0$, $(P', v) \in E$, and
\[
    \forall (P'', \_) \in E \text{ such that } P'' \neq P' :
\]
\[
    \matchlength(P, P'') \leq \matchlength(P, P')
\]
\end{document}
