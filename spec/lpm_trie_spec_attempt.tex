\documentclass{article}
\usepackage{amsmath}
\usepackage{amssymb}
\DeclareMathOperator{\matchlength}{matchLength}
\DeclareMathOperator{\triecond}{trieCondition}
\DeclareMathOperator{\initialize}{initialize}
\DeclareMathOperator{\update}{update}
\DeclareMathOperator{\delete}{delete}
\DeclareMathOperator{\lookup}{lookup}
\begin{document}
\section{Data structures}
\subsection{Node}
A node consists of a string of bits called a prefix, an integer value
associated with that prefix and two children nodes. It is represented
as follows:
\[
    Node: N(N_l, P, v, N_r)
\]
where $P = p_1p_2...$ is the prefix, $v$ is a non-negative integer and $N_l$ and
$N_r$ are the children nodes. A node can also be empty, in which case it is
represented by the special symbol $nil$. A node without any children would then
be represented as $N(nil, P, v, nil)$.
\subsection{Trie}
A trie consists of a node called the root, an integer representing the
number of nodes currently in the trie and another integer representing the
maximum number of nodes that the trie can contain. It is represented as follows:
\[
    Trie: T(R, n, m)
\]
where R is the root, n is the current number of entries and m is the maximum
number of entries.
\section{Functions}
\subsection{Preliminary definitions}
\subsubsection{Trie membership}
A node is part of a trie if there exists a path from the root to that node.
\[
    N \in T(R, n, \_) \Leftrightarrow \exists RN_1N_2...N_jN
\]
such that: $j \leq n-2$,
\[
    R(N_1, \_, \_, \_) \text{ or } R(\_, \_, \_, N_1),
\]
\[
    \forall i \in [2, j]: N_{i-1}(N_i, \_, \_, \_) \text{ or } N_{i-1}(\_, \_, \_, N_i)
\]
and
\[
    N_j(N, \_, \_, \_) \text{ or } N_j(\_, \_, \_, N)
\].
\subsubsection{Match length}
For the purpose of clarifying the trie specification, we introduce the
$\matchlength$ function. $\matchlength$ takes a node and a prefix as argument and
returns the number of bits in the prefix given as argument that match the
bits in the node's prefix. It is defined as follows:
\[
    \matchlength(N(\_, P_N, \_, \_), P) = l \in \mathbb{N}
\]
where $l$ is the largest integer such that $p_{N1}p_{N2}...p_{Nl} = p_1p_2...p_l$
and $l \leq min(|P_N|, |P|)$.
\subsubsection{Trie condition}
To preserve the useful structure of the trie, any node's children have to match
the prefix stored in their parent entirely, and the first bit beyond the
parent's prefix has to be 0 for the left child and 1 for the right child.
This is expressed by the $\triecond$ function, which is defined as follows.
\[
    \triecond(T) =
\]
\[
    \forall N(N_l, P, \_, N_r) \in T: \matchlength(N_l, P) = \matchlength(N_r, P) = |P|
\]
\[
    \text{ and } N_l(\_, P_l, \_, \_), p_{l|P|} = 0, \text{ } N_r(\_, P_r, \_, \_), p_{r|P|} = 1
\]
\section{Functions}
\subsection{initialize}
The $\initialize$ function takes an integer as argument and returns an empty trie
with the specified max number of entries. It is specified as follows:
\[
    \initialize(m) = T(nil, 0, m)
\]
where $m \in \mathbb{N} \setminus \{0\}$.
\subsection{update}
The $\update$ function takes a trie and a prefix-value pair as arguments and
inserts a new node in the trie containing the specified prefix and value. If
there is already a node corresponding to the specified prefix in the trie, then
the value stored in the node is updated. There must be space for at least one
more node in the trie given as argument. The $\update$ function must not modify
or delete any node that does not correspond to the prefix given as argument. It
is specified as follows:
\[
    \update(T(R, n, m), P, v) = T'(R, n', m)
\]
where $n < m$, $\triecond(T)$, $\triecond(T')$, $N(\_, P, v, \_) \in T'$,
\[
    \forall N' \neq N: N' \in T \Leftrightarrow N' \in T'
\]
and
\[
    n'=
    \begin{cases}
    n+1,   & \text{if } N(\_, P, \_, \_) \notin E\\
    n,  & \text{otherwise}
    \end{cases}
\]
\subsection{delete}
The $\delete$ function takes a trie and a prefix as argument and deletes the
node corresponding to the given prefix from the trie. There has to be a node in
the trie corresponding to the given prefix. The $\delete$ function must not
modify or delete any node that does not correspond to the prefix given as
argument. It is specified as follows:
\[
    \delete(T(R, n, m), P) = T'(R, n', m)
\]
where n' = n-1, $\triecond(T)$, $\triecond(T')$,
\[
    \forall N(\_, P', \_, \_) \in T': P' \neq P \text{ and }
    N \in T' \Leftrightarrow N \in T
\]
\subsection{lookup}
$\lookup$ is the main function used to perform the actual longest prefix match.
It takes as argument a non-empty trie and a prefix and returns the value stored
in the node that best matches the prefix given as argument. It is specified as
follows:
\[
    \lookup(T(R, n, \_), P) = v
\]
where $n > 0$, $v \geq 0$, $\triecond(T)$, $N(\_, \_, v, \_) \in T$ and
\[
    \forall N'\in T, N' \neq N: \matchlength(N', P) \leq \matchlength(N, P)
\]
\end{document}
