\documentclass{article}
\usepackage{amsmath}
\usepackage{amssymb}
\DeclareMathOperator{\matchlength}{matchLength}
\DeclareMathOperator{\triecond}{trieCondition}
\DeclareMathOperator{\initialize}{initialize}
\DeclareMathOperator{\update}{update}
\DeclareMathOperator{\delete}{delete}
\DeclareMathOperator{\lookup}{lookup}
\usepackage{qtree}
\begin{document}
\section{Introduction}
LPM is a common problem with applications in many areas, most notably it is at the
core of IP routers, using it to determine which entry of their routing tables should
be used to direct incoming packets. In this context, verifying an LPM implementation
would be a part of the verification process of many applications. The aim of this
work is to provide a verified LPM implementation to be used as a building block of
verified applications, using the verifast program verifier for c.
\section{Choice of implementation}
The first part of this work was to decide on an LPM implementation that would best
suit the goal that had been set out. It needed to be reasonably efficient while staying
flexible enough to fit in a wide range of applications. Another concern was to find
an implementation that would be reasonably easy to reason about in formal terms, for
the purpose of verification. The two candidates considered were the dir-24-8 LPM
table, used as a basis for the DPDK implementation of LPM and the binary-tree (trie)
LPM used by the Linux kernel.
\subsection{dir-24-8}
This implementation relies on two tables, one containing $2^{24}$ entries and storing
all entries in the table corresponding to route prefixes no more than 24 bits long
and another table storing the rest of the entries. When a new prefix is added, the
corresponding entry is stored in the table at all indexes matching this prefix. This
allows for fast lookups as the prefix given as argument can be used directly as an
index in the table, reducing the lookup operation to one or two memory accesses.
However, it also means that adding a new entry in the table will often entail updating
a significant number of entries in the table, especially if the number of entries
already present is low. In summary, dir-24-8 trades high memory use and slow
addition/deletion of entries for faster lookups.
\subsection{Trie LPM}
The Trie LPM consists of a binary tree whose nodes each store a prefix and the value
associated to them. Every node's successors match its prefix completely and are
stored on the left or the right of the node, depending on the value of the first bit
that doesn't match the node's prefix. This means that lookups are reduced to a tree
traversal, each additional level of depth in the tree providing a better match until
the best match is found. The same is true of the addition/deletion of entries, where
finding the part of the tree to be modified makes up most of the work. All operations
on the trie are therefore roughly similar in terms of performance.
\subsection{Comparison}
The first important difference to note between the dir-24-8 and trie implementations
resides in the fact that the first provides one efficient operation at the cost of
two other inefficient operations, whereas the second provides balance between the
performance of different operations. This is because dir-24-8 was specifically
designed to be used in IP routers, a context in which lookups are by far the most
common operation. An implementation heavily focused on optimizing one specific
operation doesn't fit our goal of a general purpose LPM, and the trie implementation
is still fairly efficient, as the complexity of its operations is of the order of
the depth of the tree. \par
Another key difference between the two implementations is that dir-24-8 is limited
to prefixes no longer than 32 bits, due to its very specific structure. Expanding
the implementation would dramatically increase the size of the table and thus make
the addition/deletion operations even slower, while using long prefixes in the trie
implementation would have a much lesser impact on the depth of the tree, the relevant metric. \par
In summary, the dir-24-8 implementation is only efficient under the assumption that
lookups are going to be performed much more often than additions/deletions and doesn't
scale as well as the trie implementation with long prefixes. This makes the trie
implementation better as a general purpose LPM.
\section{Changes to the original implementation}
Once the choice of an implementation was made, some work was required to get rid of
all the dependencies of the linux kernel implementation and to make it better suited
for verification. This mainly meant removing small optimizations used by the original
implementation to make the details of the data structures and the code itself clearer
to verifast and closer to the representation that would be provided to it. \par
An important change that was made to the original implementation was the pre-allocation
of all the memory necessary to the trie, which meant using the double-chain allocator
to manage memory. This change made it easier to show memory safety, as all the
memory needed was only allocated once and then only accessed throughout the code
itself. The main downside to this change was to make it impossible to properly free
the trie data structure, as it now contained the double-chain allocator, which could
itself no be freed cleanly.
\section{Trie LPM specification}
Once the trie implementation had been adapted to run outside of the linux kernel and
made more suitable for verification, the next important step was to provide a specification
for it. A pseudo-formal specification was written first in English and then translated
into the verifast language. The first version of the specification is found below.
\subsection{Data structures}
\subsubsection{Node}
A node consists of a string of bits called a prefix, an integer value
associated with that prefix and two children nodes. It is represented
as follows:
\[
    Node: N(N_l, P, v, N_r)
\]
where $P = p_1p_2...$ is the prefix, $v$ is an optional non-negative integer and
$N_l$ and $N_r$ are the children nodes. A node can also be empty, in which case
it is represented by the special symbol $nil$. A node without any children would
then be represented as $N(nil, P, v, nil)$.
\subsubsection{Trie}
A trie consists of a node called the root, an integer representing the
number of nodes currently in the trie and another integer representing the
maximum number of nodes that the trie can contain. It is represented as follows:
\[
    Trie: T(R, n, m)
\]
where R is the root, n is the current number of entries and m is the maximum
number of entries.
\subsection{Preliminary definitions}
\subsubsection{Trie membership}
A node is part of a trie if there exists a path from the root to that node.
\[
    N \in T(R, n, \_) \Leftrightarrow \exists RN_1N_2...N_jN
\]
such that: $j \leq n-2$,
\[
    R(N_1, \_, \_, \_) \text{ or } R(\_, \_, \_, N_1),
\]
\[
    \forall i \in [2, j]: N_{i-1}(N_i, \_, \_, \_) \text{ or } N_{i-1}(\_, \_, \_, N_i)
\]
and
\[
    N_j(N, \_, \_, \_) \text{ or } N_j(\_, \_, \_, N)
\].
\subsubsection{Match length}
For the purpose of clarifying the trie specification, we introduce the
$\matchlength$ function. $\matchlength$ takes a node and a prefix as argument and
returns the number of bits in the prefix given as argument that match the
bits in the node's prefix. It is defined as follows:
\[
    \matchlength(N(\_, P_N, \_, \_), P) = l \in \mathbb{N}
\]
where $l$ is the largest integer such that $p_{N1}p_{N2}...p_{Nl} = p_1p_2...p_l$
and $l \leq min(|P_N|, |P|)$.
\subsubsection{Trie condition}
To preserve the useful structure of the trie, any node's children have to match
the prefix stored in their parent entirely, and the first bit beyond the
parent's prefix has to be 0 for the left child and 1 for the right child.
This is expressed by the $\triecond$ function, which is defined as follows.
\[
    \triecond(T) =
\]
\[
    \forall N(N_l, P, \_, N_r) \in T: \matchlength(N_l, P) = \matchlength(N_r, P) = |P|
\]
\[
    \text{ and } N_l(\_, P_l, \_, \_), p_{l|P|} = 0, \text{ } N_r(\_, P_r, \_, \_), p_{r|P|} = 1
\]
\subsection{Functions}
\subsubsection{initialize}
The $\initialize$ function takes an integer as argument and returns an empty trie
with the specified max number of entries. It is specified as follows:
\[
    \initialize(m) = T(nil, 0, m)
\]
where $m \in \mathbb{N} \setminus \{0\}$.
\subsubsection{update}
The $\update$ function takes a trie and a prefix-value pair as arguments and
inserts a new node in the trie containing the specified prefix and value. If
there is already a node corresponding to the specified prefix in the trie, then
the value stored in the node is updated. There must be space for at least one
more node in the trie given as argument. The $\update$ function must not modify
or delete any node that does not correspond to the prefix given as argument. It
is specified as follows:
\[
    \update(T(R, n, m), P, v) = T'(R, n', m)
\]
where $n < m$, $\triecond(T)$, $\triecond(T')$, $N(\_, P, v, \_) \in T'$,
\[
    \forall N' \neq N: N' \in T \Leftrightarrow N' \in T'
\]
and
\[
    n'=
    \begin{cases}
    n+1,   & \text{if } N(\_, P, \_, \_) \notin E\\
    n,  & \text{otherwise}
    \end{cases}
\]
\subsubsection{delete}
The $\delete$ function takes a trie and a prefix as argument and deletes the
node corresponding to the given prefix from the trie. There has to be a node in
the trie corresponding to the given prefix. The $\delete$ function must not
modify or delete any node that does not correspond to the prefix given as
argument. It is specified as follows:
\[
    \delete(T(R, n, m), P) = T'(R, n', m)
\]
where n' = n-1, $\triecond(T)$, $\triecond(T')$,
\[
    \forall N(\_, P', \_, \_) \in T': P' \neq P \text{ and }
    N \in T' \Leftrightarrow N \in T
\]
\subsubsection{lookup}
$\lookup$ is the main function used to perform the actual longest prefix match.
It takes as argument a non-empty trie and a prefix and returns the value stored
in the node that best matches the prefix given as argument. It is specified as
follows:
\[
    \lookup(T(R, n, \_), P) = v
\]
where $n > 0$, $v \geq 0$, $\triecond(T)$, $N(\_, \_, v, \_) \in T$ and
\[
    \forall N'\in T, N' \neq N: \matchlength(N', P) \leq \matchlength(N, P)
\]
\section{Implementation details}
The update and delete functions are the most complicated and interesting parts of
the trie implementation, as the lookup function itself is only a rather trivial
tree traversal. The simplicity of lookup relies on the useful structure of the
tree, which update and delete both need to preserve. Details of the way this is
achieved are provided below.
\subsection{Update}
The first step in the update function is a tree traversal, the aim of which is to
find the position that the new node should take in the tree. This traversal can
have to distinct outcomes, either a perfect match for the prefix of the new node
is found, indicating that the node we are adding is already part of the tree. If
a full match is not found, the best match found the by the tree traversal is the
prefix of the node that will be the parent of the new node in the tree. \par
If the node we are adding is already part of the tree, there is no need to add
another node and the existing node is simply updated with the value provided
as argument to update.
\bigbreak
Initial tree: \Tree [.0/1,1 {001/3,2} {010/3,3} ]
update(010/3, 4): \Tree [.0/1,1 {001/3,2} {010/3,4} ]
\bigbreak
If the node we are adding is not part of the tree, there are again two possible
scenarios. If the parent doesn't have a child at the position where the new node
should be added, then the new node can just be added as a new child.
\bigbreak
Initial tree: \Tree [.0/1,1 {001/3,2} {010/3,3} ]
update(0010/4, 4): \Tree [.0/1,1 [.001/3,2 {0010/4,4} ] {010/3,3} ]
\bigbreak
However, it
is possible that there is already a node at the position where the new one should
be added. In this case, the position of the existing child in the tree needs to
modified. If the existing child fully matches the prefix of the new node, then
the new node is inserted between the parent and the existing child.
\bigbreak
Initial tree: \Tree [.0/1,1 {001/3,2} {010/3,3} ]
update(00/2, 4): \Tree [.0/1,1 [.00/2,4 {001/3,2} ] {010/3,3} ]
\bigbreak
If it does
not, then a new node needs to be inserted as the parent's child to differentiate
between the new node and the existing child. This new node has the shortest possible
prefix that allows for the existing child and the new node to be its children. The
nodes inserted that way are called intermediary nodes and do not store any value.
They are not considered as part of the trie by the lookup function.
\bigbreak
Initial tree: \Tree [.0/1,1 {001/3,2} {010/3,3} ]
\bigbreak
update(000/3, 4): \Tree [.0/1,1 [.00/2,IM {000/3,4} {001/3,2} ] {010/3,3} ]

\subsection{Delete}
The delete function begins as update, with a tree traversal, with the difference
that this time a full match needs to be found. Indeed, the node we are deleting
needs to be part of the tree. The traversal also identifies the parent of the node
to be deleted and its grand-parent, if they exist.
Once the position of the node to be deleted has been found and its ancestors
identified, four different cases can arise. \par
If the node we are deleting has two children, it is necessary that it stays in
the tree to differentiate between its two children. Indeed, if this was not the
case, then one would have been inserted as the child of the other. To still remove
the node from the entries taken into account by the lookup operation, it is marked
as intermediary.
\bigbreak
Initial tree: \Tree [.0/1,1 [.00/2,2 {000/3,3}  {001/3,4} ] ]
delete(00/2): \Tree [.0/1,1 [.00/2,IM {000/3,3}  {001/3,4} ] ]
\bigbreak
If the node to be removed has only one child, then that child is simply promoted
up the tree and the node can be deleted safely.
\bigbreak
Initial tree: \Tree [.0/1,1 [.00/2,2 {000/3,3} ] ]
delete(00/2): \Tree [.0/1,1 [.000/3,3 ] ]
\bigbreak
The case where the node we are deleting has no children seems trivial, nothing
needs to be done apart from removing it.
\bigbreak
Initial tree: \Tree [.0/1,1 [.00/2,2 ] ]
delete(00/2): \Tree [.0/1,1 ]
\bigbreak
However, if its parent happens to be an
intermediary node, it would not serve any purpose once the deletion is completed.
It is therefore removed from the tree as well and its other child, which we know
exists, is promoted up the tree to take the place of the intermediary node.
\bigbreak
Initial tree: \Tree [.0/1,1 [.00/2,IM {000/3,3}  {001/3,4} ] ]
delete(000/3): \Tree [.0/1,1 {001/3,4} ]
\bigbreak
\end{document}
